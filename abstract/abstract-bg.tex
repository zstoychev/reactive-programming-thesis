\documentclass[a4paper, 12pt]{article}

\usepackage{titling}
\usepackage{fontspec}
\usepackage{polyglossia}
\usepackage[backend=bibtex, style=numeric, sorting=none, language=british]{biblatex}
\usepackage[vmargin=1in, hmargin=1.20in]{geometry}

\setmainfont[Ligatures=TeX]{Liberation Serif}
\setsansfont{Liberation Sans}
\setmonofont{Liberation Mono}

\setmainlanguage{bulgarian}
\setotherlanguage{english}

\setlength{\droptitle}{-10em}

\title{Използване на принципите на реактивното и функционалното програмиране за изграждане на скалируеми и устойчиви софтуерни приложения}
\author{Здравко Стойчев Стойчев — ФН М23673}
\date{12 октомври 2015 г.}

\begin{document}
  \pagenumbering{gobble}
  \maketitle
  
  \begin{center}
    \large Резюме
    \vspace{0.2em}
  \end{center}

  В последните години броят на Интернет потребителите и свързаните в Интернет устройства расте експоненциално, а с това натовареността към софтуерните системи и изискванията към тях се увеличават многократно. Най-популярните в днешно време сред разработчиците софтуерни архитектури от своя страна се базират на монолитен модел със синхронна комуникация и строга консистентност. Системите, базирани на тези архитектури, не се справят с това натоварване. Те не оползотворяват ефективно ресурсите, които са им предоставени, скалират трудно върху многоядрени и дистрибутирани системи и често биват събаряни от неочаквани грешки.
  
  Тази дипломна работа разглежда различен модел — този на реактивното програмиране, чиято цял е да се справи с тези проблеми. Работата започва с обстоен преглед на монолитните архитектури, представяйки техните проблеми, както на концептуално, така и на имплементационно ниво. Тези проблеми биват съпоставени с принципите на реактивното програмиране и начините по които те по-добре моделират физическия свят чрез приемане на неговата същност на недетерминизъм и липса на тотална консистентност. Крайната цел на тези системи е отзивчивост към своите потребители, дори в случаите на грешки в някои техни части. Същевременно се разглежда функционалното програмиране и езиковите и концептуалните средства, които то предоставя за ограничаване на този недерминизъм и сложността, породена от него, чрез използване на неизменяеми данни и силно композиращи се абстракции.
  
  Начините на имплементация на реактивни системи е чрез комбиниране на различни конкретни средства и подходи. Това са например шаблоните реактор и проактор, \textit{\textenglish{future}} и \textit{\textenglish{promise}}, актьорският модел, различни реализации на поточни данни и други. Всички те биват анализирани подробно за това как спазват реактивните принципи, кои части от едно реактивно приложение могат да подпомогнат, как се комбинират помежду си, както и как могат да бъдат реализирани ефективно, заедно с функционалните принципи. За някои от тях се предоставят и конкретни реализации.
  
  Накрая работата предлага конкретна архитектура, реализирана чрез разгледаните реактивни средства, която е подходяща за до средно големи приложения. Тя предоставя възможност за използване скалируеми и устойчиви домейн агрегати и евентуално консистентни изгледи върху данните. След това архитектурата бива реализирана чрез конкретно примерно скалируемо и устойчиво приложение, представящо нейните различни характеристики.
\end{document} 
